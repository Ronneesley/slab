\documentclass{article}
\usepackage{geometry}
\usepackage{tabularx}
\usepackage{array}
\usepackage{multirow}
\usepackage{makecell}
\usepackage{titlesec}
\usepackage{setspace}
\usepackage{titletoc}
\usepackage{caption}
\usepackage{graphicx}
\usepackage{enumitem}
\usepackage{bm}


\geometry{a4paper, margin=1in}
\renewcommand{\arraystretch}{1.5}

% Personalização do cabeçalho
\titleformat{\section}[block]{\normalfont\Large\bfseries\filcenter}{\thesection}{1em}{}
\titlespacing*{\section}{0pt}{*2}{*2}

\renewcommand{\contentsname}{\hfill\bfseries\Large Sumário\hfill}

% Configuração para centralizar as entradas do sumário
\titlecontents{chapter}[1em]{\vspace{1ex}}{\bfseries\Large\contentslabel{2em}\hfill}{\bfseries\Large\hfill}{\bfseries\Large\hfill\contentspage}

\captionsetup[table]{name=Tabela}





\begin{document}

\title{INSTITUTO FEDERAL DE EDUCAÇÃO, CIÊNCIA E TECNOLOGIA GOIANO (IF)\\ CAMPUS CERES\\ BACHARELADO EM SISTEMAS DE INFORMAÇÃO\\ANÁLISE DE SISTEMAS ORIENTADOS A OBJETOS}
\doublespacing
\date{}
\maketitle
\vspace{50pt}

\section*{Aprimoramento do Slab utilizando Docker e documentação Caso de Uso de Calcular Média}
\vspace{70pt}

\begin{center}
\large
\section*{Autores} 
 GABRIELLA TAVARES PEIXOTO\\
 ISABELA FERNANDA RODRIGUES OLIVEIRA\\
 WANESSA MARTINS ROCHA\\
 \vspace{90pt}
 
{CERES\\2023}
\vspace{70pt}

\tableofcontents
\end{center}
\vspace{500pt}

\section{Introdução}
{Este documento aborda a documentação referente ao aprimoramento do SLab por meio da integração do Docker, fornecendo informações cruciais para compreender e implementar essa funcionalidade inovadora. A proposta visa otimizar a escalabilidade, segurança, confiabilidade, desempenho, monitoramento e limpeza do ambiente, destacando requisitos funcionais e não funcionais essenciais para o sucesso do projeto.}
\vspace{10pt}

\subsection{Requisitos}
\vspace{10pt}

\subsubsection{Requisitos não Funcionais}
{O projeto estabelece requisitos não funcionais cruciais, como a escalabilidade dos contêineres, implementação de segurança no contêiner MySQL, garantia de confiabilidade, otimização de desempenho e estabelecimento de um sistema robusto de monitoramento. Além disso, a limpeza pós-instalação é enfatizada para reduzir o tamanho da imagem final.}
\vspace{10pt}

\begin{center}
\large
\doublespacing
\captionof{table}{Requisito Não Funcional (RNF1) - Os contêineres devem ser escaláveis e gerenciados de acordo com as necessidades da aplicação.}
\begin{tabular}{|c|p{10cm}|}
    \hline
    \textbf{Identificação do requisito} & RNF1 \\
    \hline
    \textbf{Nome do Requisito} & Escalabilidade\\
    \hline
    \textbf{Local} & IF Goiano Ceres \\
    \hline
    \textbf{Data} & 08 de outubro de 2023 \\
    \hline
    \textbf{Responsável pelo Requisito} & Gabriella, Isabela e Wanessa \\
    \hline
    \multicolumn{2}{|c|}{\textbf{Especificação do Requisito}} \\
    \hline
    \multicolumn{2}{|c|}{\begin{tabular}{@{}p{10cm}@{}}Os contêineres devem ser escaláveis e gerenciados de acordo com as necessidades da aplicação. A escalabilidade permite alocação eficiente de recursos de hardware. Assim como também possibilita ter vários contêineres executando a aplicação e caso um falhe, os outros podem assumir a carga de trabalho.\end{tabular}} \\
    \hline
\end{tabular}
\end{center}
\vspace{60pt}

\begin{center}
\large
\onehalfspacing
\captionof{table}{Requisito não Funcional (RNF2) - Deve haver mecanismos de segurança implementados para proteger os dados no contêiner MySQL.}
\begin{tabular}{|c|p{10cm}|}
	\hline
	\textbf{Identificação} & RNF2 \\
	\hline
	\textbf{Nome do Requisito} & Segurança\\
	\hline
	\textbf{Local} & IF Goiano Ceres \\
	\hline
	\textbf{Data} & 08 de outubro de 2023 \\
	\hline
	\textbf{Responsável pelo Requisito} & Gabriella, Isabela e Wanessa\\
	\hline
	\multicolumn{2}{|c|}{\textbf{Especificação do Requisito}} \\
	\hline
	\multicolumn{2}{|c|}{\begin{tabular}{@{}p{10cm}@{}}O acesso seguro ao banco de dados restrito a usuários autorizados desenvolvedores do sistema, ou seja, usuários comuns não possuem acesso a alteração do banco. Os dados serão devidamente criptografados para conexões seguras com o banco de dados. Backups regulares dos dados MySQL para garantir a recuperação dos dados sem perdas significativas. Além disso, a aplicação de práticas recomendadas de segurança de contêineres deve ser instaurada.  \end{tabular}} \\
	\hline
\end{tabular}
\end{center}
\vspace{20pt}

\begin{center}
\large
\onehalfspacing
\captionof{table}{Requisito não Funcional (RNF3) - Os serviços devem ser configurados de forma que a aplicação seja altamente disponível e confiável.}
\begin{tabular}{|c|p{10cm}|}
	\hline
	\textbf{Identificação} & RNF3 \\
	\hline
	\textbf{Nome do Requisito} & Confiabilidade \\
	\hline
	\textbf{Local} & IF Goiano Ceres \\
	\hline
	\textbf{Data} & 08 de outubro de 2023 \\
	\hline
	\textbf{Responsável pelo Requisito} & Gabriella, Isabela e Wanessa\\
	\hline
	\multicolumn{2}{|c|}{\textbf{Especificação do Requisito}} \\
	\hline
	\multicolumn{2}{|c|}{\begin{tabular}{@{}p{10cm}@{}} Poderá ser realizado o balanceamento de carga, distribuindo o tráfego entre vários servidores para evitar a sobrecarga em um único ponto.\end{tabular}} \\
	\hline
\end{tabular}
\end{center}
\vspace{130pt}

\begin{center}
\large
\onehalfspacing
\captionof{table}{Requisito não Funcional (RNF4) - Os contêineres devem ser otimizados para o desempenho, garantindo que a aplicação funcione sem problemas.}
\begin{tabular}{|c|p{10cm}|}
	\hline
	\textbf{Identificação} & RNF4 \\
	\hline
	\textbf{Nome do Requisito} & Desempenho\\
	\hline
	\textbf{Local} & IF Goiano Ceres \\
	\hline
	\textbf{Data} & 08 de outubro de 2023 \\
	\hline
	\textbf{Responsável pelo Requisito} & Gabriella, Isabela e Wanessa \\
	\hline
	\multicolumn{2}{|c|}{\textbf{Especificação do Requisito}} \\
	\hline
	\multicolumn{2}{|c|}{\begin{tabular}{@{}p{10cm}@{}} Para que haja um maior desempenho, os recursos poderão ser alocados de forma eficiente nos contêineres, como CPU, memória e armazenamento, garantindo que cada contêiner tenha o necessário para funcionar. A imagem utilizada no contêiner é mais leve e otimizada, reduzindo assim o tempo de inicialização e consumo de recursos. 
 \end{tabular}} \\
	\hline
\end{tabular}
\end{center}
\vspace{20pt}

\begin{center}
\large
\onehalfspacing
\captionof{table}{Requisito não Funcional (RNF5) - Deve estabelecer um sistema de monitoramento e registro para acompanhar de perto o desempenho e os eventos dos contêineres.}
\begin{tabular}{|c|p{10cm}|}
	\hline
	\textbf{Identificação} & RNF5 \\
	\hline
	\textbf{Nome do Requisito} & Monitoramento\\
	\hline
	\textbf{Local} & IF Goiano Ceres \\
	\hline
	\textbf{Data} & 08 de outubro de 2023 \\
	\hline
	\textbf{Responsável pelo Requisito} & Gabriella, Isabela e Wanessa\\
	\hline
	\multicolumn{2}{|c|}{\textbf{Especificação do Requisito}} \\
	\hline
	\multicolumn{2}{|c|}{\begin{tabular}{@{}p{10cm}@{}} A implementação desse sistema de monitoramento e registro é crucial para garantir a estabilidade e eficiência do ambiente. Ao acompanhar o desempenho e os eventos dos contêineres, será possível tomar ações preventivas, identificar gargalos e otimizar a infraestrutura, assegurando um funcionamento contínuo e eficaz do ambiente. \end{tabular}} \\
	\hline
\end{tabular}
\end{center}
\vspace{20pt}

\begin{center}
\large
\onehalfspacing
\captionof{table}{Requisito não Funcional (RNF6) - Além de realizar a limpeza após a instalação de pacotes para reduzir o tamanho da imagem final, é crucial efetuar uma limpeza minuciosa e completa.}
\begin{tabular}{|c|p{10cm}|}
	\hline
	\textbf{Identificação} & RNF6 \\
	\hline
	\textbf{Nome do Requisito} & Limpeza\\
	\hline
	\textbf{Local} & IF Goiano Ceres \\
	\hline
	\textbf{Data} & 08 de outubro de 2023 \\
	\hline
	\textbf{Responsável pelo Requisito} & Gabriella, Isabela e Wanessa\\
	\hline
	\multicolumn{2}{|c|}{\textbf{Especificação do Requisito}} \\
	\hline
	\multicolumn{2}{|c|}{\begin{tabular}{@{}p{10cm}@{}} Envolve a remoção de caches e arquivos temporários gerados durante o processo de instalação para otimizar o espaço em disco e minimizar o tamanho da imagem resultante. \end{tabular}} \\
	\hline
\end{tabular}
\end{center}
\vspace{10pt}

\begin{center}
\large
\onehalfspacing
\captionof{table}{Requisito não Funcional (RNF7) - Preparar um ambiente ágil e otimizado para a hospedagem de uma aplicação web.}
\begin{tabular}{|c|p{10cm}|}
	\hline
	\textbf{Identificação} & RNF7 \\
	\hline
	\textbf{Nome do Requisito} & Aplicação Web\\
	\hline
	\textbf{Local} & IF Goiano Ceres \\
	\hline
	\textbf{Data} & 08 de outubro de 2023 \\
	\hline
	\textbf{Responsável pelo Requisito} & Gabriella, Isabela e Wanessa \\
	\hline
	\multicolumn{2}{|c|}{\textbf{Especificação do Requisito}} \\
	\hline
	\multicolumn{2}{|c|}{\begin{tabular}{@{}p{10cm}@{}} A integração precisa e eficaz do PHP, incluindo a gestão de dependências pelo Composer, é crucial para garantir um desempenho sem intercorrências. Além disso, a configuração adequada do ambiente Node.js é fundamental para assegurar a execução eficiente das aplicações baseadas nessa tecnologia. Garantir a harmonia entre essas diferentes tecnologias é a chave para criar um ambiente sólido e eficaz para a aplicação web. 
 \end{tabular}} \\
	\hline
\end{tabular}
\end{center}
\vspace{20pt}

\subsubsection{Requisitos funcionais}
{Os requisitos funcionais abrangem desde a configuração dos serviços do contêiner PHP e MySQL até a instalação de dependências, configuração de permissões adequadas, criação de usuários específicos, instalação de dependências do projeto e configuração do fuso horário. Esses elementos são fundamentais para assegurar a consistência, segurança e eficiência do ambiente.}
\begin{center}
\large
\onehalfspacing
\captionof{table}{Requisito Funcional (RF1) -  O serviço do contêiner PHP deve ser configurado para utilizar a imagem como base para a construção do contêiner.}
\begin{tabular}{|c|p{10cm}|}
	\hline
	\textbf{Identificação} & RF1 \\
	\hline
	\textbf{Nome do Requisito} &  Serviço do Contêiner PHP \\
	\hline
	\textbf{Local} & IF Goiano Ceres \\
	\hline
	\textbf{Data} & 08 de outubro de 2023 \\
	\hline
	\textbf{Responsável pelo Requisito} & Gabriella, Isabela e Wanessa\\
	\hline
	\multicolumn{2}{|c|}{\textbf{Especificação do Requisito}} \\
	\hline
	\multicolumn{2}{|c|}{\begin{tabular}{@{}p{10cm}@{}} Essa imagem é essencial para estabelecer o ambiente PHP dentro do contêiner. \end{tabular}} \\
	\hline
\end{tabular}
\end{center}

\begin{center}
\large
\onehalfspacing
\captionof{table}{Requisito Funcional (RF2) - O serviço do contêiner MySQL deve ser configurado para construir o contêiner com base na imagem "slab mysql".}
\begin{tabular}{|c|p{10cm}|}
	\hline
	\textbf{Identificação} & RF2 \\
	\hline
	\textbf{Nome do Requisito} & Serviço do Contêiner MySQL\\
	\hline
	\textbf{Local} & IF Goiano Ceres \\
	\hline
	\textbf{Data} & 08 de outubro de 2023 \\
	\hline
	\textbf{Responsável pelo Requisito}& Gabriella, Isabela e Wanessa \\
	\hline
	\multicolumn{2}{|c|}{\textbf{Especificação do Requisito}} \\
	\hline
	\multicolumn{2}{|c|}{\begin{tabular}{@{}p{10cm}@{}}Essas configurações são essenciais para a construção e configuração do serviço do contêiner MySQL, permitindo a conexão com o banco de dados e a comunicação com outros serviços ou aplicações externas.
\end{tabular}} \\
	\hline
\end{tabular}
\end{center}
\vspace{90pt}


\begin{center}
\large
\onehalfspacing
\captionof{table}{Requisito Funcional (RF3) - O Dockerfile destinado ao ambiente de produção, "Dockerfile ubuntu prod", é configurado para utilizar a imagem base ubuntu:22.04.}
\begin{tabular}{|c|p{10cm}|}
	\hline
	\textbf{Identificação} & RF3 \\
	\hline
	\textbf{Nome do Requisito} & Sistema Base (Ubuntu 22.04) \\
	\hline
	\textbf{Local} & IF Goiano Ceres \\
	\hline
	\textbf{Data} & 08 de outubro de 2023 \\
	\hline
	\textbf{Responsável pelo Requisito} & Gabriella, Isabela e Wanessa \\
	\hline
	\multicolumn{2}{|c|}{\textbf{Especificação do Requisito}} \\
	\hline
	\multicolumn{2}{|c|}{\begin{tabular}{@{}p{10cm}@{}} Essa escolha específica da versão do sistema operacional Ubuntu como base do contêiner é fundamental para garantir a consistência e estabilidade do ambiente de produção, sendo  essencial para assegurar a padronização e compatibilidade do ambiente, contribuindo para a estabilidade e eficiência no ciclo de vida do software produzido \end{tabular}} \\
	\hline
\end{tabular}
\end{center}
\vspace{30pt}

\begin{center}
\large
\onehalfspacing
\captionof{table}{Requisito Funcional (RF4) - Instalação de Pacotes cruciais para o funcionamento adequado.}
\begin{tabular}{|c|p{10cm}|}
	\hline
	\textbf{Identificação} & RF4 \\
	\hline
	\textbf{Nome do Requisito} & Instalação de Pacotes\\
	\hline
	\textbf{Local} & IF Goiano Ceres \\
	\hline
	\textbf{Data} & 08 de outubro de 2023 \\
	\hline
	\textbf{Responsável pelo Requisito} & Gabriella, Isabela e Wanessa \\
	\hline
	\multicolumn{2}{|c|}{\textbf{Especificação do Requisito}} \\
	\hline
	\multicolumn{2}{|c|}{\begin{tabular}{@{}p{10cm}@{}} Dockerfile deve ser configurado para atualizar o sistema e instalar uma lista específica de pacotes essenciais para o ambiente de execução específicas.\end{tabular}} \\
	\hline
\end{tabular}
\end{center}
\vspace{30pt}

\begin{center}
\large
\onehalfspacing
\captionof{table}{Requisito Funcional (RF5) - O Dockerfile para o ambiente de produção, deve ser configurado para estabelecer permissões adequadas em diretórios-chave para garantir a segurança e funcionalidade do sistema}
\begin{tabular}{|c|p{10cm}|}
	\hline
	\textbf{Identificação} & RF5 \\
	\hline
	\textbf{Nome do Requisito} & Permissões de Diretório\\
	\hline
	\textbf{Local} & IF Goiano Ceres \\
	\hline
	\textbf{Data} & 08 de outubro de 2023 \\
	\hline
	\textbf{Responsável pelo Requisito} & Gabriella, Isabela e Wanessa \\
	\hline
	\multicolumn{2}{|c|}{\textbf{Especificação do Requisito}} \\
	\hline
	\multicolumn{2}{|c|}{\begin{tabular}{@{}p{10cm}@{}}Essa criação de usuário tem o objetivo de garantir a segregação de permissões e limitar privilégios, visando à segurança e à organização da estrutura do sistema.

\end{tabular}} \\
	\hline
\end{tabular}
\end{center}
\vspace{30pt}


\begin{center}

\large
\onehalfspacing
\captionof{table}{Requisito Funcional (RF6) - O Dockerfile destinado ao ambiente de produção, deve ser configurado para criar um usuário específico chamado 'composer' com permissões apropriadas em diretórios-chave.}
\begin{tabular}{|c|p{10cm}|}
	\hline
	\textbf{Identificação} & RF6\\
	\hline
	\textbf{Nome do Requisito} & Criação de Usuário Composer\\
	\hline
	\textbf{Local} & IF Goiano Ceres \\
	\hline
	\textbf{Data} & 08 de outubro de 2023 \\
	\hline
	\textbf{Responsável pelo Requisito} & Gabriella, Isabela e Wanessa \\
	\hline
	\multicolumn{2}{|c|}{\textbf{Especificação do Requisito}} \\
	\hline
	\multicolumn{2}{|c|}{\begin{tabular}{@{}p{10cm}@{}}Deve ser adicionada uma busca por nome para que o usuário possa buscar seus dados no ranking de forma específica.\end{tabular}} \\
	\hline
\end{tabular}
\end{center}
\vspace{50pt}

\begin{center}
\large
\onehalfspacing
\captionof{table}{Requisito Funcional (RF7) - Instalação das dependências do projeto usando o Composer.}
\begin{tabular}{|c|p{10cm}|}
	\hline
	\textbf{Identificação} & RF7 \\
	\hline
	\textbf{Nome do Requisito} & Instalação de Dependências Composer\\
	\hline
	\textbf{Local} & IF Goiano Ceres \\
	\hline
	\textbf{Data} & 08 de outubro de 2023 \\
	\hline
	\textbf{Responsável pelo Requisito} & Gabriella, Isabela e Wanessa \\
	\hline
	\multicolumn{2}{|c|}{\textbf{Especificação do Requisito}} \\
	\hline
	\multicolumn{2}{|c|}{\begin{tabular}{@{}p{10cm}@{}} É necessário realizar a instalação das dependências do projeto usando o Composer, um gerenciador de dependências para PHP. Para garantir a instalação adequada e segura das bibliotecas e pacotes necessários para o funcionamento correto da aplicação.\end{tabular}} \\
	\hline
\end{tabular}
\end{center}
\vspace{30pt}

\begin{center}
\large
\onehalfspacing
\captionof{table}{Requisito Funcional (RF8) - É fundamental para garantir a consistência no registro de eventos, logs e em todas as funcionalidades sensíveis ao tempo no ambiente de desenvolvimento.}
\begin{tabular}{|c|p{10cm}|}
	\hline
	\textbf{Identificação} & RF8 \\
	\hline
	\textbf{Nome do Requisito} & Configuração do Fuso Horário\\
	\hline
	\textbf{Local} & IF Goiano Ceres \\
	\hline
	\textbf{Data} & 08 de outubro de 2023 \\
	\hline
	\textbf{Responsável pelo Requisito} & Gabriella, Isabela e Wanessa \\
	\hline
	\multicolumn{2}{|c|}{\textbf{Especificação do Requisito}} \\
	\hline
	\multicolumn{2}{|c|}{\begin{tabular}{@{}p{10cm}@{}} A configuração adequada do fuso horário é crucial para coordenar tarefas, registros e operações dentro do contexto temporal correto, especialmente em ambientes distribuídos ou sistemas que dependem de marcações temporais precisas.
\end{tabular}} \\
	\hline
\end{tabular}
\end{center}
\vspace{30pt}

\begin{center}
\large
\onehalfspacing
\captionof{table}{Requisito Funcional (RF9) - O Dockerfile deve incluir etapas para a remoção do arquivo index.html localizado no diretório /var/www/html.}
\begin{tabular}{|c|p{10cm}|}
	\hline
	\textbf{Identificação} & RF9 \\
	\hline
	\textbf{Nome do Requisito} & Remoção de index.html\\
	\hline
	\textbf{Local} & IF Goiano Ceres \\
	\hline
	\textbf{Data} & 08 de outubro de 2023 \\
	\hline
	\textbf{Responsável pelo Requisito} & Gabriella, Isabela e Wanessa \\
	\hline
	\multicolumn{2}{|c|}{\textbf{Especificação do Requisito}} \\
	\hline
	\multicolumn{2}{|c|}{\begin{tabular}{@{}p{10cm}@{}} Essa ação é necessária para viabilizar a substituição do conteúdo padrão por arquivos e dados pertinentes à aplicação em desenvolvimento. Remover o arquivo index.html permite a preparação desse diretório para receber e exibir o conteúdo específico da aplicação, assegurando que o servidor web apresente corretamente o aplicativo em construção, e não o conteúdo padrão, ao ser acessado.
\end{tabular}} \\
	\hline
\end{tabular}
\end{center}
\vspace{30pt}

\subsection{Casos de Uso}
{Este caso de uso destaca a interação do usuário logado que deseja realizar cálculos estatísticos, como a média. O sistema garante uma experiência intuitiva, onde o usuário insere valores, o sistema realiza o cálculo e exibe o resultado. Possíveis extensões, como inserção de valores inexistentes, são consideradas para uma experiência do usuário robusta.}

\subsubsection{UC}
\begin{description}[font=\normalfont\bfseries\boldmath, left=2em]
    \item[Identificador:] UC1
    \item[Nome:]  Calcular Média
    \item[Ator principal:] Usuário Logado
    \item[Interessados e Interesses:] Usuário Logado: Quem já possui uma conta no sistema e deseja realizar cálculos estatísticos, como a média.
    \item[Pré-condições:] O usuário está autenticado no sistema.
    \item[Garantia de Sucesso (Pós-condições):] O sistema exibe o resultado do cálculo da média para o conjunto de valores inseridos pelo usuário.
    \item[Cenário de Sucesso Principal (ou Fluxo Básico):]
    \begin{enumerate} 
        \item O usuário logado acessa o menu e seleciona a página “Calcule”, em seguida, escolhe a opção "Média". 
        \item O sistema exibe a tela para cálculo de média e um campo para inserção dos valores.
        \item O usuário insere os valores desejados no campo.
        \item Ao acionar o botão “Calcular”, o sistema realiza o cálculo da média.
        \item O sistema exibe o resultado da média ao usuário.
    \end{enumerate}
    \item[Extensões (ou Fluxos Alternativos):]
    \begin{itemize}
        \item Se o usuário tentar registrar um valor inexistente, o sistema exibe uma mensagem de erro.
    \end{itemize}
\end{description}
\vspace{30pt}

\end{document}

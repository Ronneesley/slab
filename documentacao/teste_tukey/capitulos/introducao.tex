\chapter{Introdução}
\label{capitulo:introducao}

O projeto propõe o desenvolvimento de um novo módulo, no Sistema SLAB, o Teste de Tukey, uma ferramenta estatística essencial, que é amplamente utilizada para comparar múltiplas médias, tornando-se valioso em investigações científicas e aplicações práticas. Este documento descreve o projeto de implementação do Teste de Tukey em uma plataforma educacional e prática denominada SLAB. Embora o SLAB seja uma fonte abrangente de conceitos estatísticos, ele não incluiu, até o momento, uma funcionalidade para calcular o Teste de Tukey. Portanto, o objetivo deste projeto foi introduzir essa funcionalidade, tornando o SLAB uma ferramenta mais completa e versátil para análise estatística.

\section{Requisitos}

%%%%%%%%%%%%%%%%
% preciso revisar
%%%%%%%%%%%%%%%%

\subsection{Requisitos Não Funcionais}

%Será que preciso mesmo disso?
\requisitoNaoFuncional{RNF1}{O sistema deve ser feito na linguagem Java}
{Dr. Fulano de Tal}{30 de outubro de 2023}{IF Goiano Ceres}
{Ronneesley Moura Teles}{rnf:linguagem}
{
Tendo em vista o \textit{stack} de tecnologia utilizado na empresa X,
optamos por utilizar a linguagem Java, já que, 90\% dos
sistemas da empresa são feitos nesta linguagem.
}

\requisitoNaoFuncional{RNF2}{O sistema deve ser portável}
{Dr. Fulano de Tal}{30 de outubro de 2023}{IF Goiano Ceres}
{Ronneesley Moura Teles}{rnf:linguagem}
{
Grande parte das máquinas da instituição funcionam em Windows e em
Linux, então o sistema deve rodar em ambos ambientes.
}

\requisitoNaoFuncional{RNF3}{O sistema deve funcionar na Web}
{Dr. Fulano de Tal}{30 de outubro de 2023}{IF Goiano Ceres}
{Ronneesley Moura Teles}{rnf:linguagem}
{
O sistema deve ser acessível via navegador Web
}

\subsection{Requisitos Funcionais}

\requisito{RF1}{O sistema deve ter um cadastro de usuários}
{Dr. Fulano de Tal}{30 de outubro de 2023}{IF Goiano Ceres}
{Ronneesley Moura Teles}{req:cadastro_usuario}
{usuario = @id + nome + cpf + email + telefone}
{
O sistema deve prover um cadastro de usuários contendo: nome,
Cadastro de Pessoa Física (CPF), e-mail e telefone.
Os usuários são as pessoas responsáveis pela venda no sistema.
}

\section{Caso de Uso}

\casoUsoDetalhadoTextual{UC1}{Fazer login}
{Internauta}
{Usuário: o usuário deseja autenticar-se para acessar as funções do sistema.}
{Nenhuma}
{O usuário estará autenticado e com uma sessão no sistema}
{
	\begin{enumerate}[label=FB\arabic*.]
		\item O internauta acessa o sistema via navegador;
		
		\item O sistema exibe a tela de login para o internauta;
		
		\item O internauta preenche o seu e-mail e senha;
		
		\item O internauta aciona o botão "Entrar";
		
		\item O sistema verifica os dados da autenticação,
		se o e-mail e senha correspondem ao de um usuário, 
		o sistema inicia uma sessão para o usuário;
		
		\item O sistema exibe a tela principal do SLab.
	\end{enumerate}
}{
	\begin{enumerate}[label=FA\arabic*.]
		\item Se o e-mail e senha não correspondem ao de um usuário:
			\begin{enumerate}
				\item O sistema volta para tela de login;
				
				\item O sistema exibe a mensagem "Usuário ou senha não encontrados"
			\end{enumerate}
	\end{enumerate}
}{}{}

\section{Classes do Sistema}

\section{Protótipos do sistema}